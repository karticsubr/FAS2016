%!TEX root = main.tex
%
% Start with sampling patterns 
%
%
\section{Assessing sampling patterns based on their spectra}
Fourier domain has been extensively used to assess the quality of different point sample distributions. 
Fourier tools like \emph{periodogram} and \emph{radial averaged periodogram}, are used since the mid 80s 
to gain an in depth knowledge of various sampling patterns. \emph{Periodogram} contains the amplitude squared 
values of Fourier coefficients that provide a good visual aid on how the low and high frequencies should be 
interpreted in a given point sample distribution. 

Following the seminal work by Robert Ulichney on \emph{blue noise} dithering, sampling patterns with minimal energy in the low frequency zone of their power spectra are advocated by researchers for applications like stippling and in rendering as well. 
Periodograms does help in investigating low frequency zone region of sampling power spectra but 
it is not always possible to get a precise behaviour of periodograms for various sampling patterns like 
jitter and Poisson Disk. This is where radially averaged power spectra comes 
handy. By performing radial averaging of sampling power spectra we can get a closer and better look at the low frequency as well as the mid to high frequency regions of the sampling power spectra.
