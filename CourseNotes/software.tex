
%----------------------------------------------------------------------------------------
 %	CHAPTER 6
 %----------------------------------------------------------------------------------------
 \chapterimage{ch6.pdf} % Chapter heading image
 \chapter{Software package}
In this chapter we explain the software toolkit provided in this course for empirical error analysis. The tool may be used to:
\begin{itemize}
 \item obtain visualizations of point-sets generated by a sampling pattern (sec.~\ref{sec:ptanalyzer});
 \item visualize Fourier spectra (sec.~\ref{sec:Fanalyzer}); 
 \item compare errors across Monte Carlo estimators that use different sampling strategies (sec.~\ref{sec:erranalyzer}); 
\end{itemize}

\section{The C++ tool}
\subsection{Set up (Ubuntu)}
The source code may be downloaded from the git repository \url{git:} \KARTIC{put repo here}.
 To compile the code, create a build subdirectory. Next, run cmake and make within the build directory. The bash commands are shown below. 
\begin{tcolorbox}
  git clone --recursive  git:// \\
  cd NAME OF REPO\\
  mkdir build\\
  cd build\\
  cmake ../ \\
  make\\
  cd ..\\
\end{tcolorbox}
The binary for empirical error analysis, `eea', is located in the build directory. The `--recursive' argument automatically clones submodules that are dependencies of the toolkit. Currently the dependencies are: \KARTIC{List dependencies.} Executing the binary without any arguments will display a list of options that need to be specified. The command line arguments for eea are divided into four sections in the following order:
 %
\begin{tcolorbox}
 ./build/eea -S  ... -I ... -A ... -G ...
\tcblower
Here, \\
-S precedes the type of samples to be used and associated parameters. \\
-I precedes the integrand type and its associated parameters \\
-A precedes analyzer name along with associated parameters \\
-G is a general section.~e.~g.~to specify output filenames.
\end{tcolorbox}
 %

\subsection{Visualizing point sets}
Point analyzer doesn't need an integrand substring (\cdash{I} $\cdots$).Each analyzer can have different set of subflags (starting with \cdashs). \
For example, the PointAnalyzer have the following:
%
\begin{tcolorbox}
./build/eea \cdash{S} \cdashs{stype} stratified \cdash{A} \cdashs{atype} pts \cdashs{nsamps} 1024 \cdashs{nreps} 1 \cdash{G} \cdashs{ofile} points
\tcblower
Here, \\
\cdashs{stype}: sampler type (stratified, random, etc.). \\
\cdashs{atype}:  Analyzer type (pts) \\
\cdashs{nsamps}: Number of samples (1024) \\
\cdashs{nreps}: Number of trials or realizations (10) \\
\cdashs{ofile}: Prefix for your output filename (pointset)
\end{tcolorbox}
%

\subsection{Visualizing Fourier spectra}
 
 %
Fourier analyzer doesn't need an integrand substring (\cdash{I} $\cdots$). Each analyzer can have different set of subflags (starting with \cdashs). \
For example, the FourierAnalyzer have the following:
%
\begin{tcolorbox}
./build/eea \cdash{S} \cdashs{stype} stratified \cdash{A} \cdashs{atype} fourier \cdashs{nsamps} 4096 \cdashs{nreps} 1 \cdashs{tstep}  2 \cdashs{wstep} 1 \cdash{G} \cdashs{ofile} powerspectrum
\tcblower
Here, \\
\cdashs{stype}: sampler type (stratified, random, etc.). \\
\cdashs{atype}:  Analyzer type (fourier) \\
\cdashs{nsamps}: Number of samples (4096) \\
\cdashs{nreps}: Number of trials or realizations (10) \\
\cdashs{tstep}:  Output intermediate result after given number of trials  \\
\cdashs{wstep}: Frequency step which is 1 for integer frequencies (it can be fractional)\\
\cdashs{ofile}: Prefix for your output filename (could be anything)
\end{tcolorbox}
%


\subsection{Analyzing error}

\section{MATLAB scripts}