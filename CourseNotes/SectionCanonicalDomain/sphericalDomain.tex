%!TEX root = ./../main.tex
%
% Start with sampling patterns 
%
%
\subsection{Spherical domain}
On the sphere, the most commonly used tool for spectral analysis is spherical harmonics
(SH)~\cite{Groemer1996}, which is analogous to Fourier analog on the sphere. While the bases for Fourier analysis are chosen to be Eigenfunctions of the derivative operator on the plane, the SH bases are chosen to be Eigenfunctions of the rotation invariant differential operator, or Laplacian, on the sphere and are
%
\begin{equation}
\label{eq:SH}
Y_l^m(\theta,\phi) \equiv {\sqrt{(2-\delta_{0m})\frac{(2l+1)}{\Lebesgue{\SphericalDom{2}}}\frac{(l-m)!}{(l+m)!}}} \, P_l^m(\cos\theta) \, \exp({im\phi})\,.
\end{equation}
%
Here, $\delta_{ij}$ is the Kronecker delta function, $\Lebesgue{\SphericalDom{2}}=4\pi$, is the Lebesgue measure of a unit 
sphere, $Y_l^m(\theta,\phi)$ is the spherical harmonic basis
function of degree $l$ and order $m$ and $P_l^m(x)$ denotes  the \emph{asssociated Legendre
Polynomials}, for $x \in [-1,1]$. SH $(Y_l^m)$ are orthonormal basis functions such that  any integrable function
$\DummyIntegrand$ on $\SphericalDom{2}$ can be decomposed into SH components.
 as:
 %
\begin{align}
\label{eq:SHDecomposition}
\DummyIntegrand(\SpaceVar) = \sum_{l=0}^{\infty} \sum_{m=-l}^{l}  \SPH{\DummyIntegrand}(l,m) \, Y_l^m(\SpaceVar)\,,
\end{align}
%
where $\SPH{\DummyIntegrand}(l,m)$ are the (complex) spectral coefficients of $\DummyIntegrand(\SpaceVar)$. 
It can be easily shown that:
%
\begin{equation}
%\label{eq:SHParsevalF}
\int_{\SphericalDom{2}} \Norm{\DummyIntegrand(\SpaceVar)} \Diff\omega = \sum_{l=0}^{\infty} \sum_{m=-l}^{l} \Norm{\SPH{\DummyIntegrand}(l,m)}\,.
\end{equation}

which is the Parseval's theorem on the sphere. Analogously,
the inner product between any two arbitrary functions, $\DummyIntegrand(\SpaceVar)$ and
$\Integrand(\SpaceVar)$ defined over a unit sphere, is related to their corresponding spectral coefficients by:
%
\begin{equation}
%\label{eq:SHParsevalGF}
\int_{\SphericalDom{2}} \DummyIntegrand(\SpaceVar) \, \mathrm{\overline{\Integrand}}(\SpaceVar) \, \Diff\SpaceVar =
\sum_{l=0}^{\infty} \sum_{m=-l}^{l}\, \SPH{\DummyIntegrand}(l,m)\cdot\overline{\SPH{\Integrand}(l,m)}\,,
\end{equation}
%
where $\SPH{\DummyIntegrand}(l,m)=\langle\DummyIntegrand,Y_l^m\rangle$ is the $(l,m)$-th spherical harmonic coefficients of $\DummyIntegrand(\SpaceVar)$. The angular mean power spectrum at a frequency band $l$ is defined as the average energy distributed over different $m$ for a given $l$, as follows:
%
\begin{equation}
%\label{eq:angular_power}
\RadialPowerSpec{\DummyIntegrand}(l) \EqDef \frac{1}{2l+1} \sum_{m=-l}^{l} \left\|\SPH{\DummyIntegrand}(l,m)\right\|^2\,.
\end{equation}
%
The \emph{mean} angular power spectrum $\RadialPowerSpec{\DummyIntegrand}(l)$ is invariant 
under a rotation of the coordinate system (\cite{Kaula1967,lowes1974spatial}), as they contain a sum over all orders $m$. 
A spherical harmonic power spectrum can be defined with or without the averaging factor. In our formulation, 
we prefer to work with the average power per degree, or power spectral density, $\RadialPowerSpec{\DummyIntegrand}(l)$, as 
this ensures that the spectral coefficients of a spherical Dirac delta function are constant and independent of degree $l$  (\cite{hipkin2001statistics}). 

%
Based on this background, variance expression for Monte Carlo integration can be derived~\cite{Pilleboue:2015:VAM} in the spherical domain as:
%
\begin{align}
\label{eq:sphere_var_finale}
\Variance{\RvMCE} = \frac{\Lebesgue{\SphericalDom{2}}}{N} \sum_{l=0}^{\infty}\, (2l+1) \Exp{\PowerSpec{\RvSampling}(l)} \RadialPowerSpec{\Integrand}(l)
\end{align}
%
which gives the final expression for the variance in terms of the angular mean power spectra of both the sampling pattern 
$\RvSampling$ and the integrand $\Integrand$. 
